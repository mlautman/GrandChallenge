\documentclass{article}
\usepackage{graphicx}
\usepackage{color}
\usepackage{listings}
\usepackage{mcode}
\usepackage{fullpage}
\usepackage{hyperref}
\usepackage{amsmath}
\usepackage{blindtext}

\definecolor{lightgray}{gray}{0.5}
\setlength{\parindent}{0pt}

\begin{document}    

	\begin{par}
	
		\title{BE 537 - Grand Challenge 1}		
		\date{\today}
		\author{James Wang, Michael Lautman, Shreel Vijayvargiya}
		\maketitle
	
	\end{par}

	\begin{par}
		\section*{Introduction}
		For this project we explored methods for performing groupwise registration between a set of brain images. We performed registration on a set of 3D brain images solving for transformations from the set of images into a common reference space. The quality of a registration was established by measuring how closely a set of labeled features in the images corresponded when projected into the common frame via the transformations we built.
	\end{par}
	
	\begin{par}
		\section*{3.1 The Basic Component}
		\subsection*{3.1.1 Extend myView to Display Registration Results}
			Our project uses a visualizer that takes in a fixed image, a moving image, the voxel spacing in the image, a rotation matrix and a translation vector. 
			\begin{lstlisting}
function myViewAffineReg(fixed, moving, spacing, A, b)
%%%%%%%%%%%%%%%%%%%%%%%%%%%%%%%%%%%%%%%%%%%%%%%%%%
%%% myView extended to display affine transformation results %%%
%%%%%%%%%%%%%%%%%%%%%%%%%%%%%%%%%%%%%%%%%%%%%%%%%%
% 	inputs 
%		fixed - fixed image                                     
%		moving - moving image                                   
%		spacing - voxel spacing                                 
%		A - 3 x 3 rotation, scaling and shearing matrix         
%		b - 3 x 1 translation matrix                            
%%%%%%%%%%%%%%%%%%%%%%%%%%%%%%%%%%%%%%%%%%%%%%%%%%
			\end{lstlisting}
			
		\subsection*{3.1.2 3D Affine Registration Objective Function}
			We compute the objective function for 3D registration using the equation below.
			\begin{align*}
				E(A,b) = \int_{\Omega} [ I(x) - J(Ax+b)]^{2} dx
			\end{align*}

			\begin{lstlisting}
function [E,g] = myAffineObjective3D(p,I,J,varargin)
%%%%%%%%%%%%%%%%%%%%%%%%%%%%%%%%%%%%%%%%%%%%%%%%%%
%%% Objective function for 3D Affine Transform  %%%
%%%%%%%%%%%%%%%%%%%%%%%%%%%%%%%%%%%%%%%%%%%%%%%%%%
% inputs
% 	p - 12 x 1 parameter vector                     
% 	I - fixed image                                 
% 	J - moving image                                
% 	varargin 
% 		dJ/dy - gradient of moving image in y direction 
% 		dJ/dx - gradient of moving image in x direction 
% 		dJ/dz - gradient of moving image in z direction 
% outputs                                 
% 	E - value of the objective function             
% 	g - gradient of the objective function          
%%%%%%%%%%%%%%%%%%%%%%%%%%%%%%%%%%%%%%%%%%%%%%%%%%
			\end{lstlisting}

		\subsection*{3.1.3 Testing the Correctness of Gradient Computation}
			We verify our analytic gradient computation by computing a numerical gradient aproximation that utilizes the central finite difference approximation.

			\begin{align*}
				\frac{\partial E}{\partial p_{j}} |_{p} \simeq  \frac{E(p + \epsilon e_{j}) - E(p - \epsilon e_{j})}{2 \epsilon}
			\end{align*}
			
			\lstinputlisting{../p313.m}
			The maximum relative error we found using an epsilon of $1e-4$ was $0.01\%$.
		
		\subsection*{3.1.4 Testing our objective function by registering two images}
		
		\lstinputlisting{../p314.m}
	\end{par}
	

\end{document}
    
